\chapter{Revisão Bibliográfica}

Este capítulo apresenta uma revisão exploratória da literatura sobre a utilização de técnicas modernas de geração de imagens no desenvolvimento de modelos de \textit{deep learning}.
Os trabalhos correlatos apresentados foram selecionados dos repositórios ACM Digital Library\footnote{https://dl.acm.org}, IEEE Digital Library\footnote{https://ieeexplore.ieee.org} e Science@Direct\footnote{https://www.sciencedirect.com/}, trabalhos publicados nos últimos 5 anos e que possuem um alto fator de impacto foram priorizados.

Para facilitar a leitura, este capítulo está organizado em três seções.
A seção \ref{RevContexto} apresenta o contexto histórico e o plano de fundo para o desenvolvimento dos trabalhos correlatos e desta pesquisa.
A seção \ref{RevResumos} descreve em ordem cronológica os trabalhos relacionados que foram desenvolvidos, destacando suas contribuições para a área.
Por fim, a seção \ref{RevConsideracoes} oferece as conclusões sobre a revisão de literatura apresentada.

\section{Fundamentação Teórica} \label{RevContexto}

As técnicas de aprendizado profundo (do inglês \textit{deep learning}) (DL) se estabelecem como modelos que possuem performance estado-da-arte na resolução de problemas que envolvem imagens, como problemas de classificação, detecção de objetos e segmentação.
No entanto, por sua natureza, modelos de DL precisam de grande volume de dados para atingirem alta performance \cite{hungAugmentationSmallTraining2021}, por esta razão, técnicas de enriquecimento e balanceamento de conjunto de dados são utilizadas para mitigar os efeitos de baixa capacidade de generalização e sobre-ajuste.

Após o estabelecimento das GANs como modelos capazes de gerar dados sintéticos úteis \cite{goodfellowGenerativeAdversarialNetworks2014}, diversos trabalhos foram desenvolvidos para avaliar a utilização desses dados na resolução de problemas de desbalanceamento de conjuntos de dados \cite{khoslaEnhancingPerformanceDeep2020}. Novos modelos baseados em GANs foram desenvolvidos para diversas aplicações ao longo do tempo, visando corrigir limitações das arquiteturas mais simples propostas inicialmente\cite{iglesiasSurveyGANsComputer2023}.
Em seu trabalho, \citeonline{dhariwalDiffusionModelsBeat2021} apresenta evidências de que modelos de difusão (do inglês Diffusion Models) (DM) são modelos capazes de gerar imagens mais realistas do que as GANs, abrindo caminho para uma nova classe de modelos a ser incorporada nas pesquisas de enriquecimento de bases de dados com imagens sintéticas \cite{yangDiffusionModelsComprehensive2024}.

\section{Construção de modelos de deep learning baseados em imagens sintéticas} \label{RevResumos}

\citeonline{hanCombiningNoisetoImageImagetoImage2019} propõe uma técnica de enriquecimento de dados de dois passos baseado em GANs para gerar e refinar imagens de ressonância magnética (RM) do cérebro, o cerne da abordagem está em gerar imagens por meio da estratégia \textit{noise-to-image} e refina-las com técnicas \textit{image-to-image} para obter imagens mais fidedignas, o modelo ResNet50 foi utilizado como classificador para avaliar a sua performance ao ser treinado nos diferentes conjuntos de dados, o trabalho conclui que utilizar imagens sintéticas junto de técnicas clássicas de enriquecimento de dados melhora a medida de sensibilidade do modelo.
\citeonline{zhuangFMRIDataAugmentation2019} avalia diferentes métodos de geração de imagens sintéticas como GANs e \textit{Variational Autoencoders} (VAE), os conjuntos de dados gerados de RM são submetidos aos modelos de \textit{support vector machine} (SVM) e uma rede neural profunda para classificação 3D, a conclusão do estudo é a de que a utilização de dados sintéticos é complementar a escolha do modelo para obter melhorias de performance e que a abordagem é promissora para mitigar o problema de falta de dados de RM. 
\citeonline{linMedicalDataAugmentation2019} argumenta em seu trabalho aplicado a classificação de imagens de raio-x de tórax que métodos clássicos de enriquecimento de dados são úteis, mas limitados por não melhorar a diversidade em conjuntos de dados pequenos, em sua pesquisa, realizam a aplicação do modelo AC-GAN, uma variação da GAN que adiciona um classificador auxiliar no discriminador para a geração de imagens, os autores optaram por avaliar a performance de uma rede de arquitetura VGG-16 nos conjuntos de dados sintéticos e reais, os resultados do trabalho mostram que a performance do modelo treinado com dados sintéticos é melhor nas medidas de acurácia, sensibilidade e F1, apesar de não encontrarem diferenças significativas para especificidade.
\citeonline{sedighGeneratingSyntheticMedical2019} explora a utilização de GANs para enriquecer conjuntos de dados de imagens de pele para detecção de câncer de pele, após submeter os diferentes conjuntos de dados a uma CNN os resultados mostraram que as métricas de acurácia, sensibilidade, especificidade e F1 são aprimoradas.
\citeonline{xuSemiSupervisedAttentionGuidedCycleGAN2019} propõe a arquitetura \textit{semi-supervised attention-guided CycleGAN} (SSA-CycleGAN) para ser aplicada em imagens médicas de RM, seu método permite gerar imagens realísticas de tumores, permitindo inclusive adicionar tumores malignos nas imagens originais para adicionar mais diversidade ao conjunto de dados original, os autores avaliam a performance da GAN em 3 conjuntos de dados distintos de imagens de RM e após treinar uma rede baseada na arquitetura ResNet18 obtêm resultados que apontam que a utilização dos conjuntos de dados enriquecidos melhora as métricas de sensibilidade e especificidade.

\citeonline{dimitrakopoulosISINGGANAnnotatedData2020} apresenta um novo método de enriquecimento de dados para modelos de segmentação de células em imagens de microscopia, o modelo ISING-GAN, os resultados experimentais mostraram que a métrica \textit{Intersection of Union} (IoU) dos resultados da rede de segmentação U-Net melhoram em relação a utilização de imagens geradas por arquiteturas clássicas de GANs e a não utilização de imagens sintéticas.
\citeonline{zhuDataAugmentationUsing2020a} em seu trabalho sobre classificação de vigor de plantas, estuda variações na arquitetura condicional deep convolutional generative adversarial network (cDCGAN) para gerar imagens de plantas saudáveis e não saudáveis, a utilização do conjunto de dados incluindo imagens sintéticas para treinamento resulta em um aumento significativo da performance de classificação no F1, obtendo resultados similares a treinar o mesmo modelo com conjuntos de dados maiores sem nenhum enriquecimento de imagens sintéticas.
\citeonline{luoSYNTHETICMINORITYCLASS2020} em seu trabalho sobre detecção automática de alvos em imagens de satélite utiliza o modelo \textit{progressive growing GAN} (PGGAN) proposto em \citeonline{karrasProgressiveGrowingGANs2018} para equilibrar a quantidade de imagens da classe minoritária do conjunto de dados utilizado, o trabalho conclui que realizar a sobreamostragem da classe minoritária elevou a acurácia da rede ResNet18 utilizada.
\citeonline{sasmalImprovedEndoscopicPolyp2020} avalia a utilização do modelo DCGAN na sintetização da imagens de pólipos benignos e malignos para o treinamento de um modelo classificador baseado em CNN, ao avaliar a performance do classificador treinado versus os trabalhos estado-da-arte na literatura, a análise conclui que o modelo obtém resultados similares, sendo uma opção viável para a produção de modelos com performance estado-da-arte, os autores argumentam que a utilização de dados sintéticos é interessante não por apenas aumentar o número de amostras para o treinamento, mas também por adicionar variabilidade no conjunto de dados.

\citeonline{farooqProofofConceptTechniquesGenerating2021} em seu trabalho sobre reconhecimento do sexo biológico de pessoas utilizando imagens termais, avalia a utilização do modelo StyleGAN para produzir imagens térmicas sintéticas com uma variedade maior de ângulos, estilos de cabelo e uso de acessórios, além da sintetização das imagens o modelo PRNet também é utilizado para realizar a reconstrução 3D, por fim o trabalho avalia a utilização destes conjuntos de dados para treinar uma rede neural ResNet50 em diferentes cenários de balanceamento e processamento, utilizando técnicas clássicas de enriquecimento de dados e GANs, a pesquisa conclui que os modelos treinados em dados sintéticos obtém uma acurácia maior versus o modelo treinado apenas em dados reais.
Em \citeonline{hungAugmentationSmallTraining2021} o modelo DCGAN é utilizado para adicionar variação nos conjuntos de dados originais (MNIST e \textit{rock paper scissors}) com o objetivo de avaliar o resultado na performance de diferentes redes neurais profundas como AlexNet, VGGNet, GoogLeNet e ResNet, os autores observaram que a utilização das imagens geradas pelo modelo DCGAN melhora a medida de acurácia dos classificadores.
\citeonline{ohConstructingVesselDataset2021} explora a ideia de construir um conjunto de dados sintéticos de imagens de embarcações utilizando os modelos CycleGAN e StyleGAN-2, após a construção dos conjuntos de dados um modelo com arquitetura EfficientNet-b0 foi utilizado para a tarefa de classificação multi-classe, os autores chegam a conclusão de que a abordagem tem potencial para a expansão de conjuntos de dados com baixo volume de dados, mostrando em seus resultados que a combinação de dados reais com os dados sintéticos leva a classificadores com melhor acurácia.
\citeonline{posilovicGenerativeAdversarialNetwork2021} discute e propõe em seu trabalho sobre detecção de defeitos automática em imagens de ultrassom a utilização de GANs para construir imagens de ultrassonografia com defeito sem a necessidade da realização de testes destrutivos, os trabalho propõe o modelo DetectionGAN com uma arquitetura específica para o problema discutido no artigo, o modelo de detecção de objetos YOLO é utilizado e treinado em diferentes cenários de conjuntos de dados, os resultados dos experimentos concluem que é de extrema importância que as imagens geradas pelos modelos de geração de imagens sejam próximas da realidade, os resultados obtidos com o detector de objetos treinado com dados gerados pelo modelo DetectionGAN tiveram uma precisão média maior do que usar somente dados reais, além disso o trabalho adiciona que a performance do modelo pode ser deteriorada se as imagens geradas não forem fidedignas o suficiente.
\citeonline{hwangImageDataAugmentation2021} trabalha a utilização do modelo TripleGAN para a geração de imagens sintéticas de satélite para o reconhecimento automático de alvos, a metodologia é aplicada ao conjunto de dados \textit{moving and stationary target acquisition and recognition} (MSTAR). A rede VGG16 é utilizada para a tarefa de classificação multi-classe, os resultados finais dos experimentos demonstraram que a utilização do conjunto de dados enriquecido com dados sintéticos obtém uma acurácia maior.
Em \citeonline{viertelPollenGANSyntheticPollen2021} é proposto o modelo PollenGAN para geração de imagens de pólen que podem ser utilizadas para o treinamento de modelos de classificação de tipos de pólen, o trabalho destaca que a avaliação da performance e qualidade das GANs é uma tarefa desafiadora e que há época ainda não havia um padrão-ouro a ser utilizado, por isso, a utilidade de conjunto de dados é medida pela eficácia na tarefa de aprendizado de máquina, por meio de uma rede CNN proposta pelos autores, os resultados dos experimentos mostram que as medidas de precisão e F1 tiveram resultados mistos, sendo que para algumas classes de pólen a performance do modelo era superior, mas em outras similar ou até ligeiramente pior.

\citeonline{kimGANBasedSyntheticData2022} estuda a implementação do modelo BicycleGAN para produzir imagens sintéticas de câmeras infravermelho, o arcabouço proposto além de mapear as imagens, é capaz de inserir alvos artificiais no imagem final para que seja utilizada em tarefas de detecção de objetos, os autores empregam diferentes modelos de detecção como ACM U-Net e DNA-Net, a conclusão do trabalho é a de que a utilização das imagens sintéticas melhora as métricas \textit{area under curve receiver operating characteristic} AUCROC e \textit{normalized intersection of union} (nIoU) dos modelos de detecção em vista de usar somente imagens reais.
\citeonline{liuSyntheticDataAugmentation2022} propõe o modelo \textit{Multiscale Attention CycleGAN} (MSA-CycleGAN) para geração de imagens sintéticas contendo aeronaves, com o objetivo de serem utilizadas para a tarefa de detecção em imagens ópticas de sensoriamento remoto, a inovação em seu trabalho está na inserção de modelos 3D de aeronaves nas imagens seguido pela adição do módulo MSA na GAN com para reduzir distorções de texturas e ruídos, os modelos de detecção utilizados foram as redes Faster R-CNN e R-FCN, o trabalho conclui que a utilização de imagens sintéticas em alinhamento com imagens reais é uma abordagem promissora de enriquecimento de dados, especialmente em casos de conjuntos de dados de volume limitado.
\citeonline{guoSARImageData2022} aborda a utilização de imagens sintéticas para a detecção de navios em imagens de satélite, para tal o trabalho propõe a uma nova arquitetura de GAN, chamada \textit{residual and attention-based GAN} (RAGAN), o modelo proposto trás melhorias para reduzir o problema de gradientes desvanecentes e adiciona o mecanismo de atenção para que a rede aprenda as características principais. Após a sintetização do conjunto de dados de imagens de satélite de navios, um modelo YOLO é utilizado para realizar a tarefa de detecção de objetos, os autores concluem que a utilização de dados sintéticos permite melhores resultados no modelo em relação as métricas de precisão média.
Em \citeonline{singhDataAugmentationSpectrally2022}, dados multi-espectrais de sensoriamento remoto são sintetizados a partir do modelo \textit{spectral indexed GAN} (SIGAN), a partir de um conjunto de dados composto por imagens de diferentes classes como florestas, rodovias, pastos, zonas residenciais, zonas industriais e corpos d'água, os autores sintetizam novas imagens e avaliam a performance de um classificador com arquitetura DCNN na tarefa de identificar a cobertura e uso do solo, os resultados mostram que para todas as classes, a acurácia do modelo treinado com dados sintéticos foi maior.
\citeonline{moonStudyNeRFbasedSynthetic2022} explora a utilização do modelo \textit{Neural radiance fields} (NeRF) para gerar imagens sintéticas foto realistas de diferentes pontos de vista a partir de uma imagem 2D, o propósito deste conjunto de dados é avaliar se um modelo de detecção de objetos como YOLOX teria uma performance equivalente se treinado em dados sintéticos, os autores também implementam um método de desfoque para gerar um segundo conjunto de dados sintéticos, a discussão dos resultados chega a conclusão de que a utilização de imagens sintéticas pós processadas faz com que o modelo YOLOX tenha um desempenho equivalente ao modelo treinado apenas em imagens reais.
\citeonline{liuDataAugmentationUsing2022} em seu trabalho sobre classificação de saburra lingual, propõe a utilização do modelo NICE-GAN para gerar imagens sintéticas de línguas com diferentes tipos de saburra lingual, os autores argumentam que para alguns conjuntos de dados a utilização de GANs condicionais é limitada pelo pequeno volume de imagens da classe minoritária, para contornar este problema, o trabalho propõe um modelo baseado na abordagem \textit{image-to-image} em que a partir das imagens reais pode-se produzir imagens sintéticas de diferentes classes presentes no conjunto de dados. Após a confecção do conjunto de dados o modelo de classificação CoAtNet de arquitetura autoral é avaliado em diferentes conjuntos de treinamento, o trabalho conclui que a utilização de imagens sintéticas melhora a acurácia e a medida F1 do modelo de classificação avaliado.
\citeonline{fooImageDataAugmentation2022} propõe um novo método de enriquecimento de dados mitigar o problema da variação dos diferentes tipos de câmeras de fotografia, a proposta dos autores é utilizar o modelo CycleGAN para realizar a transferência de estilos de modo que uma mesma imagem exista com diferentes padrões de captura no conjunto de dados, o trabalho mostra que a técnica pode ser aplicada para qualquer tipo conjunto de dados de imagem. Para avaliar a performance do método proposto, os autores utilizaram uma rede ResNet50 e a testaram em diferentes conjuntos de dados e configurações de enriquecimento de dados, os resultados dos experimentos mostraram que a utilização de transferência de estilo melhorou a medida de acurácia do classificador e superou também a performance dos métodos mais clássicos de enriquecimento de dados.
\citeonline{randoDCGANbasedMedicalImage2022} apresenta a utilização do modelo DCGAN em conjunto com a técnica \textit{Gray Level Co-occurrence Matrix} (GLMC) para a geração de imagens e extração de características, o objetivo do trabalho é a construção de um um conjunto de dados para posterior detecção de três diferentes classes de patologias em imagens produzidas a partir de exames de papanicolau, o desenvolvimento de um modelo de classificação usando Extreme Learning Machine (ELM) mostrou que a utilização dos dados sintéticos melhorou a performance da acurácia de classificação, sendo inclusive melhor do que usar apenas os dados originais. 

\citeonline{deptoQuantifyingImbalancedClassification2023a} investiga a utilização de diferentes técnicas de correção de desbalanceamento de dados para o treinamento de modelos de detecção de leucemia. Diferentes modelos como VGG16, ResNet50, DenseNet121 e EfficientNetb0 são submetidos ao treinamento em diferentes cenários de pré-processamento incluindo: não realizar balanceamento de classes, sobreamostragem de classe minoritária, subamostragem de classe majoritária, \textit{autoencoders} e CycleGAN, além de serem ajustados em diferentes configurações de funções de perda. O método de explicabilidade Grad-CAM também é aplicado para aprofundar o entendimento das principais características aprendidas pelos modelos e auxiliar na avaliação, o estudo conclui que a utilização de métodos baseados em função de perda são mais eficazes para produzir melhores classificadores do que a utilização de dados sintéticos na tarefa de detecção de leucemia.
\citeonline{rozanecSyntheticDataAugmentation2023} usa a \textit{Lightweight} GAN em conjunto da rede ResNet18 para lidar com o problema de inspeção visual automática para detecção de defeitos em processos de controle de qualidade. A hipótese de que os dados sintéticos em conjunto dos dados reais pode aumentar a performance discriminativa do modelo de classificação é testada e os resultados dos experimentos mostram que para problemas de classificação binária e multi-classe a utilização dos dados sintéticos produzidos por GANs foi capaz de melhorar a performance do modelo.
\citeonline{giusteExplainableSyntheticImage2023a} desenvolve um arcabouço incluindo Progressive GAN (PGAN) para geração de imagens sintéticas, Inspirational GAN (IGAN) para inclusão de patologias nas imagens e diferentes técnicas de explicabilidade como Grad-CAM++, Eigen-CAM, Score-CAM e Ablation-CAM para auxiliar na avaliação das previsões feitas pelos modelos. O trabalho teve como objetivo estudar a utilização de imagens sintéticas na detecção de rejeição de transplantes de coração por meio de imagens de microscopia. Os resultados mostraram uma melhora nas métricas dos modelos de classificação empregados e o trabalho também conclui que a utilização de técnicas de explicabilidade promove transparência no processo de tomada de decisão guiado por IA, dado são consistentes com as análises de especialistas da área.
Em \citeonline{ivanovsSyntheticImageGeneration2023} é proposto a utilização de modelos de difusão estável utilizando a abordagem \textit{text-to-image} para enriquecer um conjunto de dados de microscopia para classificação de imagens de células, utilizando um modelo EfficientNetB7 como classificador os resultados concluem que a acurácia do modelo é pior ao utilizar o conjunto de dados sintéticos, mesmo quando utilizados em conjuntos com dados reais em diferentes proporções.
\citeonline{masengiInvestigatingDeepLearning2023} investiga a viabilidade da utilização do modelo DCGAN para sintetização de imagens de RM para mitigar o problema de desbalanceamento de classes na tarefa de classificação de distúrbios psiquiátricos. Utilizando a rede VGG16, diferentes proporções de imagens sintéticas foram avaliadas para resolver o problema de desbalanceamento, os resultados dos experimentos concluíram que em nenhum dos cenários o modelo treinado com dados sintéticos foi melhor, adicionando a discussão o fato de que modelos de geração de imagens baseados em GANs ainda precisam de uma número suficiente de amostras originais para realizar um enriquecimento de dados efetivo.
\citeonline{youngchoiAutomatedDetectionCrystalline2023a} propõe a utilização de uma \textit{multistage} CycleGAN  com o objetivo de gerar imagens de fotografia de fundo para a tarefa de detecção de retinopatia cristalina. O modelo proposto adiciona nas imagens de retinas saudáveis os depósitos de cristais característicos da patologia de modo que uma diversidade maior de imagens é obtida. Os modelos ResNet50 e EfficientNetB0 são utilizados para tarefa de classificação e então o método Grad-CAM é utilizado para gerar os mapas de atenção das saídas dos modelos. Os resultados dos experimentos concluem que o modelo treinado em dados sintéticos utilizando o método \textit{multistage} CycleGAN possuem uma performance superior as técnicas tradicionais de transferência de estilo, além de aprenderem as características relevantes de acordo com os mapas de atenção gerados pelo método Grad-CAM.
\citeonline{gaoDiffGuardSemanticMismatchGuided2023} avalia a utilização de modelos de difusão para geração de instâncias sintéticas para a construção de classificadores de amostras fora da distribuição, o objetivo desses modelos é identificar quando uma instância faz ou não parte da mesma distribuição estatística do conjunto de treinamento, para lidar de maneira mais robusta com a extrapolação do modelo em cenários ainda não vistos. Os modelos de geração utilizados foram \textit{DiffGuard}, \textit{Guided Difussion Model} (GDD) e \textit{Latent Difussion Model} (LDM), enquanto os modelos de classificação construídos com base na arquitetura ResNet, os resultados concluem que a utilização dos dados sintéticos melhoram as métricas de AUCROC, taxa de falsos positivos e a sensibilidade dos modelos de classificação.
\citeonline{baoRareHeartTransplant2023} explora a utilização de modelos de difusão para geração de imagens sintéticas para detecção automática de rejeição de transplante de coração em imagens de microscopia, o trabalho propõe a utilização de \textit{Denoising Diffusion Probabilistic Model} (DDPM) e uma GAN para gerar imagens sintéticas e então avaliar a performance de um classificador EfficientNet-b3 em diferentes cenários de composições de imagens sintéticas e reais a partir dos dois modelos de geração utilizados. Os experimentos concluem que a utilização de modelos de difusão produzem um conjunto de dados mais rico e capaz de melhorar a performance de classificação do modelo, além disso o método Grad-CAM++ é empregado para avaliar os mapas de atenção dos modelos produzidos, concluindo que o classificador treinado no conjunto de dados produzido pelo modelo de difusão gera classificações baseadas em características mais relevantes que o modelo treinado em dados produzidos pela GAN.

\citeonline{xieGANBasedSubInstanceAugmentation2024} propõe um novo método de enriquecimento de dados chamado \textit{GAN-based sub-instance augmentation} (GSIA), permitindo a geração de imagens realísticas e diversas para a construção de melhores detectores de mudanças em minerações a céu-aberto. A partir da construção do conjunto de dados diferentes classificadores como FC-Siam-conc, SNUNet, BIT, A2Net e DMINet são treinados e avaliados utilizando conjuntos de dados puramente reais e sintéticos. O trabalho conclui que classificadores treinados no conjunto de dados produzido possuem performance similar aos classificadores treinados em conjuntos de dados reais.
\citeonline{eshunDeepConvolutionalNeural2024} desenvolve uma DCGAN com arquiteturas de geradores e discriminadores específicos para mitigar os problemas de instabilidade da geração de dados sintéticos para a classe minoritária, o trabalho aplica a metodologia proposta no problema de detecção de câncer de mama utilizando um classificador ResNet50. Os autores concluem que o método proposto supera os \textit{baselines} da literatura utilizando métodos clássicos de enriquecimento de dados e também os modelos treinados puramente em dados reais.

\section{Considerações} \label{RevConsideracoes}

A utilização de conjuntos de dados sintéticos para a construção de modelos de classificação de imagens recebeu muitas contribuições em diversas áreas de aplicação desde o estabelecimento dos modelos estado da arte como GANs e \textit{diffusion models}.
Na tabela \ref{tab:revisao} é oferecido ao leitor um resumo dos trabalhos previamente discutidos, muitos trabalhos propuseram e obtiveram sucesso no uso das imagens sintéticas como uma abordagem para mitigar o problema de ajustar modelos de DL em conjuntos de dados limitados, no entanto, encontra-se também na literatura referências com resultados similares \cite{xieGANBasedSubInstanceAugmentation2024} e piores \cite{ivanovsSyntheticImageGeneration2023}\cite{deptoQuantifyingImbalancedClassification2023a} \cite{masengiInvestigatingDeepLearning2023} em relação a outras técnicas de enriquecimento já existentes ou a utilização de apenas imagens reais.

\begin{table}[htbp]
\caption{Resumo dos trabalhos correlatos}
\resizebox{\textwidth}{!}{
\begin{tabular}{p{2in} p{3in} p{3in} p{2in} p{1in} p{2in}}
    \hline
    \textbf{Trabalho} &
      \textbf{Modelos de geração utilizados} &
      \textbf{Modelos de aprendizagem} &
      \textbf{Tarefa} &
      \textbf{Performance com dados sintéticos} &
      \textbf{Métodos de interpretabilidade} \\
      \hline
\citeonline{hanCombiningNoisetoImageImagetoImage2019} &
  PGGAN, SimGAN &
  ResNet50, PRNet &
  Classificação &
  Melhor &
  - \\
\citeonline{linMedicalDataAugmentation2019} &
  AC-GAN &
  VGG-16 &
  Classificação &
  Melhor &
  - \\
\citeonline{xuSemiSupervisedAttentionGuidedCycleGAN2019} &
  SSA-CycleGAN &
  ResNet18 &
  Classificação &
  Melhor &
  - \\
\citeonline{zhuangFMRIDataAugmentation2019} &
  GMM, CVAE, ICW-GAN &
  SVM, 3D deep net classifier &
  Classificação &
  Melhor &
  - \\
\citeonline{sedighGeneratingSyntheticMedical2019} &
  GAN &
  CNN &
  Classificação &
  Melhor &
  - \\
\citeonline{zhuDataAugmentationUsing2020a} &
  cDCGAN &
  ResNet &
  Classificação &
  Melhor &
  - \\
\citeonline{dimitrakopoulosISINGGANAnnotatedData2020} &
  ISING-GAN,GAN,ResGAn,ISING-ResGAN &
  U-Net &
  Segmentação &
  Melhor &
  - \\
\citeonline{luoSYNTHETICMINORITYCLASS2020} &
  PGGAN &
  ResNet18 &
  Classificação &
  Melhor &
  - \\
\citeonline{sasmalImprovedEndoscopicPolyp2020} &
  DCGAN &
  CNN &
  Classificação &
  Melhor &
  - \\
\citeonline{farooqProofofConceptTechniquesGenerating2021} &
  StyleGAN &
  ResNet50, PRNet &
  Classificação &
  Melhor &
  - \\
\citeonline{hungAugmentationSmallTraining2021} &
  DCGAN &
  AlexNet, GoogLeNet, ResNet, VGGNet &
  Classificação &
  Melhor &
  - \\
\citeonline{hwangImageDataAugmentation2021} &
  TripleGAN &
  VGG-16 &
  Classificação &
  Melhor &
  - \\
\citeonline{ohConstructingVesselDataset2021} &
  StyleGAN2-ada, CycleGAN &
  EfficientNet &
  Classificação &
  Melhor &
  - \\
\citeonline{posilovicGenerativeAdversarialNetwork2021} &
  DetectionGAN &
  YOLO &
  Detecção de objetos &
  Melhor &
  - \\
\citeonline{viertelPollenGANSyntheticPollen2021} &
  cDCGAN &
  CNN &
  Classificação &
  Melhor &
  - \\
\citeonline{fooImageDataAugmentation2022} &
  CycleGAN &
  ResNet50 &
  Classificação &
  Melhor &
  - \\
\citeonline{guoSARImageData2022} &
  RAGAN &
  YOLOv3 &
  Detecção de objetos &
  Melhor &
  - \\
\citeonline{kimGANBasedSyntheticData2022} &
  BicycleGAN &
  U-Net &
  Segmentação &
  Melhor &
  - \\
\citeonline{liuDataAugmentationUsing2022} &
  NICE-GAN &
  CoAtNet &
  Classificação &
  Melhor &
  - \\
\citeonline{liuSyntheticDataAugmentation2022} &
  MSA-CycleGAN, CycleGAN &
  R-CNN, R-FCN &
  Detecção de objetos &
  Melhor &
  - \\
\citeonline{moonStudyNeRFbasedSynthetic2022} &
  NerF-W &
  YOLOX-M &
  Detecção de objetos &
  Melhor &
  - \\
\citeonline{randoDCGANbasedMedicalImage2022} &
  DCGAN &
  Extreme Learning Machine &
  Classificação &
  Melhor &
  - \\
\citeonline{singhDataAugmentationSpectrally2022} &
  SIGAN &
  DCNN &
  Classificação &
  Melhor &
  - \\
\citeonline{deptoQuantifyingImbalancedClassification2023a} &
  CycleGAN &
  ResNet50, DenseNet121, EfficientNet &
  Classificação &
  Pior &
  Grad-CAM \\
\citeonline{giusteExplainableSyntheticImage2023a} &
  PGGAN, IGAN, Difussion Model &
  ResNet50, ResNet152, DenseNet161 &
  Classificação &
  Melhor &
  Grad-CAM++, Eigen-CAM, Score-CAM, Ablation-CAM \\
\citeonline{youngchoiAutomatedDetectionCrystalline2023a} &
  CycleGAN &
  ResNet50, EfficientNet &
  Classificação &
  Melhor &
  Grad-CAM \\
\citeonline{baoRareHeartTransplant2023} &
  DiffusionModel, PGAN &
  EfficientNet-b3 &
  Classificação &
  Melhor &
  Grad-CAM++ \\
\citeonline{gaoDiffGuardSemanticMismatchGuided2023} &
  DiffGuard, GDD, LDM &
  ResNet18, ResNet50 &
  Classificação &
  Melhor &
  - \\
\citeonline{ivanovsSyntheticImageGeneration2023} &
  LDM &
  EfficientNet B7 &
  Classificação &
  Pior &
  - \\
\citeonline{masengiInvestigatingDeepLearning2023} &
  DCGAN &
  VGG16 &
  Classificação &
  Pior &
  - \\
\citeonline{rozanecSyntheticDataAugmentation2023} &
  GAN, StyleGAN &
  ResNet18, MLP, GBT, DREAM &
  Classificação &
  Melhor &
  - \\
\citeonline{eshunDeepConvolutionalNeural2024} &
  DCGAN &
  ResNet50, VGGNet, NDCNN, VGG, CaffeNet &
  Classificação &
  Melhor &
  - \\
\citeonline{xieGANBasedSubInstanceAugmentation2024} &
  GSIA &
  FC-Siam-conc, SNUNet, BIT, A2Net, DMINet &
  Classificação &
  Melhor &
  -
      \\
      \hline
\end{tabular}
}
\label{tab:revisao}
\source{Alexandre Farias, 2024}
\end{table}

A aplicação de técnicas de explicabilidade se mostrou fundamental em alguns trabalhos para determinar se os conjuntos de imagens produzidos conseguiam capturar as características fundamentais a serem aprendidas pelos classificadores \cite{youngchoiAutomatedDetectionCrystalline2023a} \cite{baoRareHeartTransplant2023} \cite{giusteExplainableSyntheticImage2023a} \cite{deptoQuantifyingImbalancedClassification2023a}, além de demonstrarem que nem todos os conjuntos de dados sintéticos produzem os mesmos modelos de classificação, no entanto, até onde se sabe não se encontra na literatura um trabalho mais extensivo em avaliar a capacidade de modelos treinados em dados sintéticos aprenderem as mesmas características de modelos treinados em dados reais, reforçando assim a necessidade da avaliação da hipótese proposta no presente trabalho.
