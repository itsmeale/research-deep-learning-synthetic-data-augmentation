\chapter{Introdução}

% tema
% motivação
% lacuna
% hipotese [OK]
% objetivo [OK]
% metodologia [OK]


\section{Objetivo}

O objetivo deste trabalho é propor e desenvolver um arcabouço para avaliação do desempenho de modelos de \textit{deep learning} na tarefa de classificação de imagens, treinados em conjuntos de dados sintéticos e reais, levando em consideração métricas e conjuntos de dados padronizados, possibilitando que modelos treinados com conjuntos de dados reais e sintéticos - provenientes de diferentes modelos de geração de imagens - possam ser avaliados sob a luz de métricas que possibilitem a detecção de falhas de generalização e especificação.

Para este fim, técnicas de explicabilidade de modelos caixa-preta e processamento de imagens serão empregadas para o desenvolvimento de novas ferramentas que permitam avaliam com mais profundidade os modelos de classificação desenvolvidos.

\section{Hipótese}

Dado o contexto e objetivos supracitados, a hipótese do presente trabalho é de que a utilização de técnicas de explicabilidade e métodos clássicos de avaliação permitam avaliar com maior rigor se modelos de classificação treinados em dados sintetizados possuem a mesma performance e também aprendem relações semelhantes na tarefa de reconhecimento de padrões.

\section{Organização deste documento}

Para facilitar a leitura o presente trabalho está divido em capítulos, o capítulo \ref{capitulo:conceitos-fundamentais} explica os conceitos fundamentais que servem de base para o desenvolvimento deste trabalho, já o capítulo \ref{capitulo:revisao-bibliografica} discorre sobre trabalhos correlatos destacando seus métodos e resultados, por sua vez, o capítulo \ref{capitulo:estudo-piloto} apresenta resultados parciais já obtidos nesta pesquisa, por fim o capítulo \ref{capitulo:proposta} descreve em detalhes a proposta de pesquisa do presente trabalho.